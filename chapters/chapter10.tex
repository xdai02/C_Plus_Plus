\chapter{面向对象}

\section{封装}

\subsection{类与对象}

在面向对象编程中,把构成问题的事物分解成各个对象,每个对象都有自己的数据和行为,程序通过对象之间的交互来实现功能。\\

类(class)是一个模板,定义了对象的属性和方法,用来描述同一类对象的共同特征和行为。对象(object)是类的实例,它具有类定义的属性和方法。\\

关键字new实例化并返回一个指向类对象的指针,之后就可以通过访问对象的属性和方法来操作对象。使用new分配的对象,需要使用delete释放。\\

\mybox{银行账户}

\begin{lstlisting}[language=C++]
#include <iostream>
#include <string>

using namespace std;

class BankAccount {
public:
    string owner;
    string account;
    double balance;

    void deposit(double amount) {
        balance += amount;
    }

    void withdraw(double amount) {
        balance -= amount;
    }
};

int main() {
    BankAccount account;
    account.owner = "Terry";
    account.account = "6250941006528599";
    account.balance = 50;

    cout << "Owner: " << account.owner << endl;
    cout << "Account: " << account.account << endl;
    cout << "Balance: " << account.balance << endl;

    account.deposit(100);
    cout << "Balance: " << account.balance << endl;

    account.withdraw(70);
    cout << "Balance: " << account.balance << endl;

    return 0;
}
\end{lstlisting}

\begin{tcolorbox}
    \mybox{运行结果}
    \begin{verbatim}
Owner: Terry
Account: 6250941006528599
Balance: 50
Balance: 150
Balance: 80
\end{verbatim}
\end{tcolorbox}

\vspace{0.5cm}

\subsection{封装(Encapsulation)}

封装是面向对象的重要原则,尽可能隐藏对象的内部实现细节。封装可以认为是一个保护屏障,防止该类的数据被外部随意访问。当要访问该类的数据时,必须通过指定的接口。合适的封装可以让代码更容易理解和维护,也加强了程序的安全性。\\

为了实现封装,需要对类的属性和方法进行访问权限的控制:

\begin{enumerate}
    \item public:允许任何地方访问。
    \item private:只允许在类的内部访问。
    \item protected:只允许在类的内部和子类中访问。
\end{enumerate}

通常会将类的属性设置为private,然后对外提供一对setter/getter方法来访问该属性。\\

为了避免方法的参数与类的属性重名造成歧义,可以使用this关键字用来指代当前对象。\\

\mybox{银行账户}

\begin{lstlisting}[language=C++]
#include <iostream>
#include <string>

using namespace std;

class BankAccount {
    private:
    const size_t ACCOUNT_DIGITS = 16;
    string owner;
    string account;
    double balance;

    public:
    void setOwner(string owner) {
        if (!owner.empty()) {
            this->owner = owner;
        }
    }

    string getOwner() {
        return owner;
    }

    void setAccount(string account) {
        if (account.length() == ACCOUNT_DIGITS) {
            this->account = account;
        }
    }

    string getAccount() {
        return account;
    }

    void setBalance(double balance) {
        if (balance >= 0) {
            this->balance = balance;
        }
    }

    double getBalance() {
        return balance;
    }

    bool deposit(double amount) {
        if (amount <= 0) {
            return false;
        }
        balance += amount;
        return true;
    }

    bool withdraw(double amount) {
        if (amount <= 0 || amount > balance) {
            return false;
        }
        balance -= amount;
        return true;
    }
};

int main() {
    BankAccount account;
    account.setOwner("Terry");
    account.setAccount("6250941006528599");
    account.setBalance(50);

    cout << "Owner: " << account.getOwner() << endl;
    cout << "Account: " << account.getAccount() << endl;
    cout << "Balance: " << account.getBalance() << endl;

    account.deposit(100);
    cout << "Balance: " << account.getBalance() << endl;

    account.withdraw(70);
    cout << "Balance: " << account.getBalance() << endl;

    return 0;
}
\end{lstlisting}

\begin{tcolorbox}
    \mybox{运行结果}
    \begin{verbatim}
Owner: Terry
Account: 6250941006528599
Balance: 50
Balance: 150
Balance: 80
\end{verbatim}
\end{tcolorbox}

\newpage

\section{构造函数}

\subsection{构造方法(Constructor)}

构造方法是一种特殊的方法,会在创建对象时自动调用,用于创建并初始化对象。每个类可以有一个或多个构造方法,构造方法的名字必须和类名一致。构造方法没有返回值,返回值类型部分不写。

\vspace{-0.5cm}

\begin{lstlisting}[language=C++]
BankAccount() {
    owner = "admin";
    account = "0000000000000000";
    balance = 0;
}
\end{lstlisting}

如果一个类中没有写构造方法,系统会自动提供一个public的无参构造方法,以便实例化对象。如果一个类中已经写了构造方法,系统将不会再提供默认的无参构造方法。

\vspace{-0.5cm}

\begin{lstlisting}[language=C++]
BankAccount(string owner, string account, double balance) {
    if (!owner.empty()) {
        this->owner = owner;
    }

    if (account.length() == ACCOUNT_DIGITS) {
        this->account = account;
    }

    if (balance >= 0) {
        this->balance = balance;
    }
}
\end{lstlisting}

析构函数(Destructor)是一种特殊的方法,会在对象被销毁时自动调用,用于释放对象占用的资源。析构函数的名字必须和类名一致,前面加上~,没有返回值,返回值类型部分不写。

\vspace{-0.5cm}

\begin{lstlisting}[language=C++]
~BankAccount() {
    // code
}
\end{lstlisting}

\vspace{0.5cm}

\subsection{重载(Overload)}

重载用于在同一个类定义多个同名方法,但是这些方法的参数列表不同。重载的主要用途是提供方法的多种版本,以便满足不同的需求。\\

重载还可以使代码更具可读性,因为它使得方法名更具描述性,而不必考虑特定的参数列表。\\

\mybox{银行账户}

\begin{lstlisting}[language=C++]
#include <iostream>
#include <string>

using namespace std;

class BankAccount {
    private:
    const size_t ACCOUNT_DIGITS = 16;
    string owner;
    string account;
    double balance;

    public:
    BankAccount() {
        owner = "admin";
        account = "0000000000000000";
        balance = 0;
    }

    BankAccount(string owner, string account, double balance) {
        if (!owner.empty()) {
            this->owner = owner;
        }

        if (account.length() == ACCOUNT_DIGITS) {
            this->account = account;
        }

        if (balance >= 0) {
            this->balance = balance;
        }
    }

    void setOwner(string owner) {
        if (!owner.empty()) {
            this->owner = owner;
        }
    }

    string getOwner() {
        return owner;
    }

    void setAccount(string account) {
        if (account.length() == ACCOUNT_DIGITS) {
            this->account = account;
        }
    }

    string getAccount() {
        return account;
    }

    void setBalance(double balance) {
        if (balance >= 0) {
            this->balance = balance;
        }
    }

    double getBalance() {
        return balance;
    }

    bool deposit(double amount) {
        if (amount <= 0) {
            return false;
        }
        balance += amount;
        return true;
    }

    bool withdraw(double amount) {
        if (amount <= 0 || amount > balance) {
            return false;
        }
        balance -= amount;
        return true;
    }

    bool withdraw(double amount, double fee) {
        if (amount <= 0 || amount + fee > balance) {
            return false;
        }

        balance -= amount + fee;
        return true;
    }
};

int main() {
    BankAccount account1;
    cout << "Account 1 Owner: " << account1.getOwner() << endl;
    cout << "Account 1 Account: " << account1.getAccount() << endl;
    cout << "Account 1 Balance: " << account1.getBalance() << endl;

    BankAccount account2("Terry", "6250941006528599", 50);
    cout << "Account 2 Balance: " << account2.getBalance() << endl;

    account2.withdraw(20);
    cout << "Account 2 Balance: " << account2.getBalance() << endl;

    account2.withdraw(10, 1);
    cout << "Account 2 Balance: " << account2.getBalance() << endl;

    return 0;
}
\end{lstlisting}

\begin{tcolorbox}
    \mybox{运行结果}
    \begin{verbatim}
Account 1 Owner: admin
Account 1 Account: 0000000000000000
Account 1 Balance: 0
Account 2 Balance: 50
Account 2 Balance: 30
Account 2 Balance: 19
	\end{verbatim}
\end{tcolorbox}

\newpage

\section{继承}

\subsection{继承(Inheritance)}

继承指一个类可以继承另一个类的特征和行为,并可以对其进行扩展。这样就可以避免在多个类中重复定义相同的特征和行为。\\

\begin{figure}[H]
    \centering
    \begin{tikzpicture}
        \begin{class}[text width = 5cm]{Product}{0,0}
            \attribute{- name : String}
            \attribute{- price : double}
            \operation{+ Product(String, double)}
            \operation{+ getName() : String}
            \operation{+ getPrice() : double}
            \operation{+ setName(String) : void}
            \operation{+ setPrice(double) : void}
        \end{class}

        \begin{class}[text width = 6cm]{Food}{-4,-7}
            \inherit{Product}
            \attribute{- calories : int}
            \operation{+ Food(String, double, int)}
            \operation{+ getCalories() : int}
            \operation{+ setCalories(int) : void}
        \end{class}

        \begin{class}[text width = 6cm]{Drink}{4,-7}
            \inherit{Product}
            \attribute{- size : String}
            \operation{+ Drink(String, double, String)}
            \operation{+ getSize() : String}
            \operation{+ setSize(String) : void}
        \end{class}
    \end{tikzpicture}
    \caption{继承}
\end{figure}

extends关键字用于指定一个类继承于另一个类,产生继承关系后,子类可以通过super关键字调用父类中的属性和方法,也可以定义子类独有的属性和方法。\\

在创建子类对象时,会先调用父类的构造方法,然后再调用子类的构造方法。因此父类中必须存在一个构造方法,否则将无法创建子类对象。\\

\mybox{麦当劳}

\begin{lstlisting}[language=Java]
public class Product {
    private String name;
    private double price;

    public Product(String name, double price) {
        this.name = name;
        this.price = price;
    }

    public String getName() {
        return name;
    }

    public void setName(String name) {
        this.name = name;
    }

    public double getPrice() {
        return price;
    }

    public void setPrice(double price) {
        this.price = price;
    }
}
\end{lstlisting}

\begin{lstlisting}[language=Java]
public class Food extends Product {
    int calories;

    public Food(String name, double price, int calories) {
        super(name, price);
        this.calories = calories;
    }

    public int getCalories() {
        return calories;
    }

    public void setCalories(int calories) {
        this.calories = calories;
    }
}
\end{lstlisting}

\begin{lstlisting}[language=Java]
public class Drink extends Product {
    private String size;

    public Drink(String name, double price, String size) {
        super(name, price);
        this.size = size;
    }

    public String getSize() {
        return size;
    }

    public void setSize(String size) {
        this.size = size;
    }
}
\end{lstlisting}

\begin{lstlisting}[language=Java]
public class McDonalds {
    public static void main(String[] args) {
        Food food = new Food("Cheeseburger", 5.45, 302);
        Drink drink = new Drink("Coke", 3.7, "Large");

        System.out.printf(
                "Food: %s ($%.2f) %d Kcal\n",
                food.getName(), food.getPrice(), food.getCalories()
        );
        System.out.printf(
                "Drink: %s ($%.2f) %s\n",
                drink.getName(), drink.getPrice(), drink.getSize()
        );
    }
}
\end{lstlisting}

\begin{tcolorbox}
    \mybox{运行结果}
    \begin{verbatim}
Food: Cheeseburger ($5.45) 302 Kcal
Drink: Coke ($3.70) Large
	\end{verbatim}
\end{tcolorbox}

\vspace{0.5cm}

\subsection{重写(Override)}

Object类是所有类的根类,所有的类都直接或者间接地继承自Object类。Object类中包含的方法在其它所有类中都可以使用,例如getClass()、hashCode()、toString()、clone()、equals()等。\\

当直接输出一个对象时,会自动调用该对象的toString()方法,将其以字符串的形式输出。

\vspace{-0.5cm}

\begin{lstlisting}[language=Java]
System.out.println(food);   // Food@41629346
\end{lstlisting}

在没有重写toString()方法的情况下,输出的内容是对象的类名及其哈希码(hash code),但这并不是预期想要的结果。因此,可以重写从父类继承的toString(),以满足程序的需求。\\

在重写方法时,需要使用@Override注解,以便编译器检查该方法是否真的是从父类继承的。\\

Object类的equals()方法默认比较的是两个对象的地址,如果地址相同则为true,否则为false。而String类的equals()方法就重写了Object类的equals()方法,以便比较两个字符串的内容是否相同。\\

\mybox{麦当劳}

\begin{lstlisting}[language=Java]
public class Product {
    private String name;
    private double price;

    public Product(String name, double price) {
        this.name = name;
        this.price = price;
    }

    public String getName() {
        return name;
    }

    public void setName(String name) {
        this.name = name;
    }

    public double getPrice() {
        return price;
    }

    public void setPrice(double price) {
        this.price = price;
    }

    @Override
    public String toString() {
        return String.format("%s ($%.2f)", name, price);
    }
}
\end{lstlisting}

\begin{lstlisting}[language=Java]
public class Food extends Product {
    int calories;

    public Food(String name, double price, int calories) {
        super(name, price);
        this.calories = calories;
    }

    public int getCalories() {
        return calories;
    }

    public void setCalories(int calories) {
        this.calories = calories;
    }

    @Override
    public String toString() {
        return "Food: " + super.toString() + " " + calories + " Kcal";
    }
}
\end{lstlisting}

\begin{lstlisting}[language=Java]
public class Drink extends Product {
    private String size;

    public Drink(String name, double price, String size) {
        super(name, price);
        this.size = size;
    }

    public String getSize() {
        return size;
    }

    public void setSize(String size) {
        this.size = size;
    }

    @Override
    public String toString() {
        return "Drink: " + super.toString() + " " + size;
    }
}
\end{lstlisting}

\begin{lstlisting}[language=Java]
public class McDonalds {
    public static void main(String[] args) {
        Food food = new Food("Cheeseburger", 5.45, 302);
        Drink drink = new Drink("Coke", 3.7, "Large");

        System.out.println(food);
        System.out.println(drink);
    }
}
\end{lstlisting}

\begin{tcolorbox}
    \mybox{运行结果}
    \begin{verbatim}
Food: Cheeseburger ($5.45) 302 Kcal
Drink: Coke ($3.70) Large
	\end{verbatim}
\end{tcolorbox}

\newpage

\section{抽象类}

\subsection{抽象类(Abstract Class)}

有些类只能用来做继承,不能用于创建对象。例如在动物园中并不存在“动物”这个对象,只有动物的子类对象,因此动物类不应该被实例化。\\

抽象类是一种不能被实例化的类,它用于定义接口或公共实现,供其它类继承并实现。

\vspace{-0.5cm}

\begin{lstlisting}[language=Java]
public abstract class Animal {}
\end{lstlisting}

\vspace{0.5cm}

\subsection{抽象方法}

有时候父类提供的方法无法满足子类不同的需求,但是如果不定义该方法,就表示该类具有该行为。\\

这种情况就可以将这个父类的方法定义为抽象方法,这样所有的子类都必须要重写该方法,否则子类仍然为抽象类。\\

抽象方法只需声明,而不用实现。包含抽象方法的类必须声明为抽象类。\\

\mybox{动物}

\begin{lstlisting}[language=Java]
public abstract class Animal {
    public abstract String sound();
}
\end{lstlisting}

\begin{lstlisting}[language=Java]
public class Dog extends Animal {
    @Override
    public String sound() {
        return "Woof";
    }
}
\end{lstlisting}

\begin{lstlisting}[language=Java]
public class Cat extends Animal {
    @Override
    public String sound() {
        return "Meow";
    }
}
\end{lstlisting}

\begin{lstlisting}[language=Java]
public class AnimalSound {
    public static void main(String[] args) {
        Dog dog = new Dog();
        Cat cat = new Cat();

        System.out.println("Dog's sound: " + dog.sound());
        System.out.println("Cat's sound: " + cat.sound());
    }
}
\end{lstlisting}

\begin{tcolorbox}
    \mybox{运行结果}
    \begin{verbatim}
Dog's sound: Woof
Cat's sound: Meow
	\end{verbatim}
\end{tcolorbox}

\newpage

\section{多态}

\subsection{多态(Polymorphism)}

多态是指对象可以具有多种形态,即同一个对象在不同时刻表现出不同的行为。例如Dog和Cat都是Animal的子类,因此可以将子类对象赋值给父类引用,从而产生多种形态。

\vspace{-0.5cm}

\begin{lstlisting}[language=Java]
Animal animal = new Dog();
\end{lstlisting}

由子类类型转型为父类类型,称为向上转型。向上转型是一种隐式转换,向上转型后的对象,只能访问父类中定义的成员。\\

由父类类型转型为子类类型,称为向下转型。向下转型存在失败的可能,因为父类对象并不一定是子类对象。\\

向下转型需要显式地强制转换,在转换时需要使用instanceof关键字进行类型检查。\\

\mybox{员工工资}

\begin{lstlisting}[language=Java]
public abstract class Employee {
    private String name;

    public Employee(String name) {
        this.name = name;
    }

    public String getName() {
        return name;
    }

    public abstract double getSalary();
}
\end{lstlisting}

\begin{lstlisting}[language=Java]
public class FullTimeEmployee extends Employee {
    private double basicSalary;
    private double bonus;

    public FullTimeEmployee(
            String name, double basicSalary, double bonus
    ) {
        super(name);
        this.basicSalary = basicSalary;
        this.bonus = bonus;
    }

    @Override
    public double getSalary() {
        return basicSalary + bonus;
    }
}
\end{lstlisting}

\begin{lstlisting}[language=Java]
public class PartTimeEmployee extends Employee {
    private double dailyWage;
    private int workingDays;

    public PartTimeEmployee(
            String name, double dailyWage, int workingDays
    ) {
        super(name);
        this.dailyWage = dailyWage;
        this.workingDays = workingDays;
    }

    @Override
    public double getSalary() {
        return dailyWage * workingDays;
    }
}
\end{lstlisting}

\begin{lstlisting}[language=Java]
public class Salary {
    public static void main(String[] args) {
        Employee[] employees = new Employee[]{
                new FullTimeEmployee("Alice", 5783, 173),
                new PartTimeEmployee("Bob", 150, 15)
        };

        for (Employee employee : employees) {
            System.out.println(
                    employee.getName() + ": $" + employee.getSalary()
            );
        }
    }
}
\end{lstlisting}

\begin{tcolorbox}
    \mybox{运行结果}
    \begin{verbatim}
Alice: $5956.0
Bob: $2250.0
	\end{verbatim}
\end{tcolorbox}

\newpage

\section{接口}

\subsection{接口(Interface)}

接口是一种特殊的抽象类,接口中的所有方法都是抽象方法,接口中的成员变量必须是常量。\\

接口用于定义一组标准,代表了某种能力和约定。例如USB接口代表了电脑和外设设备之间的连接标准。USB接口不用关心连接的外设设备是什么,只要这个外设设备满足USB的标准,就可以连接到电脑上。\\

\mybox{USB}

\begin{lstlisting}[language=Java]
public interface USB {
    String getDeviceInfo();
}
\end{lstlisting}

\begin{lstlisting}[language=Java]
public class Mouse implements USB {
    @Override
    public String getDeviceInfo() {
        return "Mouse";
    }
}
\end{lstlisting}

\begin{lstlisting}[language=Java]
public class Keyboard implements USB {
    @Override
    public String getDeviceInfo() {
        return "Keyboard";
    }
}
\end{lstlisting}

\begin{lstlisting}[language=Java]
public class Computer {
    private USB usb1;
    private USB usb2;

    public void setUSB1(USB usb) {
        usb1 = usb;
    }

    public void setUSB2(USB usb) {
        usb2 = usb;
    }

    @Override
    public String toString() {
        return "USB 1: " + usb1.getDeviceInfo() + "\n"
                + "USB 2: " + usb2.getDeviceInfo();
    }

    public static void main(String[] args) {
        Computer computer = new Computer();

        computer.setUSB1(new Mouse());
        computer.setUSB2(new Keyboard());

        System.out.println(computer);
    }
}
\end{lstlisting}

\begin{tcolorbox}
    \mybox{运行结果}
    \begin{verbatim}
USB 1: Mouse
USB 2: Keyboard
	\end{verbatim}
\end{tcolorbox}

\newpage