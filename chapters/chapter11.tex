\chapter{C++基础}

\section{重载函数}

\subsection{函数默认参数}

在进行函数参数定义的时候,也可以设置默认值。当参数没有传递的时候就利用默认值来进行参数内容的填充,如果在参数上定义了默认值,那么该参数一定要放在参数列表的最后。\\

\mybox{函数默认参数}

\begin{lstlisting}[language=C++]
#include <iostream>

using namespace std;

void setDate(int year = 1970, int month = 1, int day = 1) {
    cout << year << "/" << month << "/" << day << endl;
}

int main() {
    setDate(2021, 8, 15);
    setDate(2021, 7);
    setDate(2021);
    setDate();
    return 0;
}
\end{lstlisting}

\begin{tcolorbox}
	\mybox{运行结果}
	\begin{verbatim}
2021/8/15
2021/7/1
2021/1/1
1970/1/1
	\end{verbatim}
\end{tcolorbox}

\vspace{0.5cm}

\subsection{重载函数}

重载(overload)表示在同一个作用域中声明了一个与之前声明过的函数具有相同名称的函数,但是它们的参数列表不同。当调用一个重载函数时,编译器通过传递的参数类型,选用最合适的定义。\\

\mybox{重载函数}

\begin{lstlisting}[language=C++]
#include <iostream>
using namespace std;

int max(int num1, int num2) {
    return num1 > num2 ? num1 : num2;
}

double max(double num1, double num2) {
    return num1 > num2 ? num1 : num2;
}

char max(char num1, char num2) {
    return num1 > num2 ? num1 : num2;
}

int main() {
    cout << max(2, 8) << endl;
    cout << max(3.14, 2.71) << endl;
    cout << max('H', 'D') << endl;
    return 0;
}
\end{lstlisting}

\begin{tcolorbox}
	\mybox{运行结果}
	\begin{verbatim}
8
3.14
H
	\end{verbatim}
\end{tcolorbox}

\subsection{new/delete}

new和delete运算符用于动态申请和释放内存空间。如果空间分配失败,程序则抛出bad\_alloc异常。

\vspace{-0.5cm}

\begin{lstlisting}[language=C++]
int *p = new int;
int *q = new int[N];

delete p;
delete[] q;
\end{lstlisting}

\vspace{0.5cm}

\begin{table}[H]
    \centering
    \setlength{\tabcolsep}{5mm}{
        \begin{tabular}{|c|c|c|}
            \hline
                              & \textbf{new/delete}      & \textbf{malloc()/free()} \\
            \hline
            \textbf{类型}     & 运算符                   & 函数                     \\
            \hline
            \textbf{分配方式} & 根据数据类型             & 根据指定大小             \\
            \hline
            \textbf{返回值}   & 返回所分配数据类型的指针 & 返回void *               \\
            \hline
            \textbf{分配失败} & 抛出bad\_alloc异常       & 返回NULL                 \\
            \hline
        \end{tabular}
    }
    \caption{new/delete与malloc()/free()的区别}
\end{table}

\newpage