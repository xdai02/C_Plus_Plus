\chapter{继承}

\section{继承}

\subsection{继承(Inheritance)}

继承是面向对象的三大特征之一,程序中的继承是类与类之间的特征和行为的一种赠予或获取。两个类之间的继承必须满足“is a”的关系。子类继承自父类,父类也称基类或超类,子类也称派生类。

\begin{figure}[H]
	\centering
	\begin{tikzpicture}
		\begin{class}{Animal}{0,0}
			\attribute{- age}
			\attribute{- gender}
			\operation{+ eat() : void}
			\operation{+ sleep() : void}
		\end{class}

		\begin{class}{Dog}{-5,-5}
			\inherit{Animal}
			\attribute{- type}
			\operation{+ bite() : void}
		\end{class}

		\begin{class}{Cat}{5,-5}
			\inherit{Animal}
			\attribute{- hairColor}
			\operation{+ mewing() : void}
		\end{class}
	\end{tikzpicture}
	\caption{继承}
\end{figure}

产生继承关系后,子类可以使用父类中的属性和方法,也可以定义子类独有的属性和方法。

\vspace{-0.5cm}

\begin{lstlisting}[language=Java]
class subclass : access_modifier superclass {
    // code
};
\end{lstlisting}

继承时通常使用public类型。当一个类public继承于父类时,父类的public成员也是子类的public成员,父类的protected成员也是子类的protected成员,父类的private成员不能被继承。\\

继承的好处是可以提高代码的复用性、提高代码的拓展性。\\

\mybox{继承}

\begin{lstlisting}[language=C++, title=animal.h]
#ifndef _ANIMAL_H_
#define _ANIMAL_H_

#include <string>

class Animal {
public:
    Animal(std::string name = "", int age = 0);
    void eat();

private:
    std::string name;
    int age;
};

#endif
\end{lstlisting}

\begin{lstlisting}[language=C++, title=animal.cpp]
#include "animal.h"
#include <iostream>

using namespace std;

Animal::Animal(string name, int age)
    : name(name), age(age) {}

void Animal::eat() {
    cout << "eating" << endl;
}
\end{lstlisting}

\begin{lstlisting}[language=C++, title=dog.h]
#ifndef _DOG_H
#define _DOG_H_

#include "animal.h"
#include <string>

class Dog : public Animal {
public:
    Dog(std::string name, int age, std::string type = "");
    void bite();
    
private:
    std::string type;
};

#endif
\end{lstlisting}

\begin{lstlisting}[language=C++, title=dog.cpp]
#include "dog.h"
#include <iostream>

using namespace std;

Dog::Dog(string name, int age, string type)
    : Animal(name, age), type(type) {}

void Dog::bite() {
    cout << "biting" << endl;
}
\end{lstlisting}

\begin{lstlisting}[language=C++, title=test\_dog.cpp]
#include <iostream>
#include "dog.h"

using namespace std;

int main() {
    Dog dog("狗子", 3, "哈士奇");
    dog.eat();
    dog.bite();
    return 0;
}
\end{lstlisting}

\begin{tcolorbox}
	\mybox{运行结果}
	\begin{verbatim}
eating
biting
	\end{verbatim}
\end{tcolorbox}

\newpage

\section{多继承}

\subsection{多继承}

C++支持多继承,即一个子类可以有两个或更多个父类。多继承时通过使用逗号将多个父类隔开,每个父类都可以用不同访问限定符修饰。\\

当多个父类中有同名的成员时,就会产生命名冲突,因此这时就需要在成员前加上类名和域限定符【::】消除二义性。\\

\mybox{多继承}

\begin{lstlisting}[language=C++, title=date.h]
#ifndef _DATE_H_
#define _DATE_H_

#include <string>

class Date {
public:
    Date(int year = 1970, int month = 1, int day = 1);
    std::string getDate();

private:
    int year;
    int month;
    int day;
};

#endif
\end{lstlisting}

\begin{lstlisting}[language=C++, title=date.cpp]
#include "date.h"

using namespace std;

Date::Date(int year, int month, int day)
    : year(year), month(month), day(day) {}

string Date::getDate() {
    char format[128];
    snprintf(format, sizeof(format), 
            "%04d/%02d/%02d", year, month, day);
    string dateStr(format);
    return dateStr;
}
\end{lstlisting}

\begin{lstlisting}[language=C++, title=time.h]
#ifndef _TIME_H_
#define _TIME_H_

#include <string>

class Time {
public:
    Time(int hour = 0, int minute = 0, int second = 0);
    std::string getTime();

private:
    int hour;
    int minute;
    int second;
};

#endif
\end{lstlisting}

\begin{lstlisting}[language=C++, title=time.cpp]
#include "time.h"

using namespace std;

Time::Time(int hour, int minute, int second)
    : hour(hour), minute(minute), second(second) {}

string Time::getTime() {
    char format[128];
    snprintf(format, sizeof(format), 
            "%02d:%02d:%02d", hour, minute, second);
    string timeStr(format);
    return timeStr;
}
\end{lstlisting}

\begin{lstlisting}[language=C++, title=date\_time.h]
#ifndef _DATE_TIME_H_
#define _DATE_TIME_H_

#include "date.h"
#include "time.h"
#include <string>

class DateTime : public Date, public Time {
public:
    DateTime(int year = 1970, int month = 1, int day = 1,
             int hour = 0, int minute = 0, int second = 0);
    std::string getDateTime();

private:
    int year;
    int month;
    int day;
    int hour;
    int minute;
    int second;
};

#endif
\end{lstlisting}

\begin{lstlisting}[language=C++, title=date\_time.cpp]
#include "date_time.h"

using namespace std;

DateTime::DateTime(int year, int month, int day,
         int hour, int minute, int second)
  : Date(year, month, day),
    Time(hour, minute, second) {}

string DateTime::getDateTime() {
    return getDate() + " " + getTime();
}
\end{lstlisting}

\begin{lstlisting}[language=C++, title=test\_date\_time.cpp]
#include <iostream>
#include "date_time.h"

using namespace std;

int main() {
    DateTime dt1;
    cout << dt1.getDateTime() << endl;
    DateTime dt2(2021, 8, 31, 13, 50, 23);
    cout << dt2.getDateTime() << endl;
    return 0;
}
\end{lstlisting}

\begin{tcolorbox}
	\mybox{运行结果}
	\begin{verbatim}
1970/01/01 00:00:00
2021/08/31 13:50:23
	\end{verbatim}
\end{tcolorbox}

\newpage

\section{向上转型与向下转型}

\subsection{向上转型 / 向下转型}

对象由子类类型转型为父类类型,即是向上转型。向上转型是一种隐式转换,一定会转型成功。向上转型后的对象,只能访问父类中定义的成员。\\

由父类类型转型转型为子类类型,即是向下转型。向下转型是不安全的,可能会导致数据的丢失,原因是父类的指针或引用中可能不包含子类成员的内存。\\

\mybox{向上转型}

\begin{lstlisting}[language=C++, title=animal.h]
#ifndef _ANIMAL_H_
#define _ANIMAL_H_

#include <string>

class Animal {
public:
    Animal(std::string name = "");
    std::string getName();

private:
    std::string name;
};

#endif
\end{lstlisting}

\begin{lstlisting}[language=C++, title=animal.cpp]
#include "animal.h"

using namespace std;

Animal::Animal(string name) : name(name) {}

string Animal::getName() {
    return name;
}
\end{lstlisting}

\begin{lstlisting}[language=C++, title=dog.h]
#ifndef _DOG_H_
#define _DOG_H_

#include "animal.h"
#include <string>

class Dog : public Animal {
public:
    Dog(std::string name, std::string type = "");
    std::string getType();

private:
    std::string type;
};

#endif
\end{lstlisting}

\begin{lstlisting}[language=C++, title=dog.cpp]
#include "dog.h"

using namespace std;

Dog::Dog(string name, string type) 
    : Animal(name), type(type) {}

string Dog::getType() {
    return type;
}
\end{lstlisting}

\begin{lstlisting}[language=C++, title=test\_dog.cpp]
#include <iostream>
#include "animal.h"
#include "dog.h"

using namespace std;

int main() {
    Dog dog("狗子", "哈士奇");
    cout << "dog: " << dog.getName()
         << ", " << dog.getType() << endl; 

    Animal animal = (Animal)dog;
    cout << "animal: " << animal.getName() << endl;
    return 0;
}
\end{lstlisting}

\begin{tcolorbox}
	\mybox{运行结果}
	\begin{verbatim}
dog: 狗子, 哈士奇
animal: 狗子
    \end{verbatim}
\end{tcolorbox}

\newpage