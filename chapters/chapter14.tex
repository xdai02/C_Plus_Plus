\subsection{虚函数}

虚函数是定义在基类中的函数,子类必须对其进行重写/覆盖(override),虚函数需要在类的成员函数前面加上virtual关键字。\\

重写/覆盖是指子类中存在重新定义的函数,其函数名、参数列表、返回值类型都与父类中被重写的函数一致。被重写的函数必须是虚函数。\\

子类若重写了父类的函数,那么子类将会隐藏其父类中被重写的函数。但是子类通过强制类型转换成父类后可以重新调用父类中被重写的函数。\\

\subsection{纯虚函数}

在虚函数后加上【= 0】后可以让这个函数变成纯虚函数,包含纯虚函数的类叫做抽象类(abstract class)或接口类(interface)。\\

抽象类不能被用于实例化对象,只是提供了所有的子类共有的部分。例如在动物园中,存在的都是动物具体的子类对象,并不存在动物对象,所以动物类不应该被独立创建成对象。\\

抽象类的作用是可以被子类继承,提供共性的属性和方法。父类提供的方法很难满足子类不同的需求,如果不定义该方法,则表示所有的子类都不具有该行为。如果定义该方法,所有的子类都在重写,那么这个方法在父类中是没有必要实现的,显得多余。\\

被virtual关键字修饰的方法称为纯虚函数。纯虚函数只有声明,没有实现。纯虚函数只能包含在抽象类中。产生继承关系后,子类必须重写父类中所有的纯虚函数,否则子类还是抽象类。\\

\newpage