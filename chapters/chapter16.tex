\chapter{I/O库}

\section{文件I/O}

\subsection{文件I/O}

程序不仅要从控制台进行I/O,还需要读写文件和字符串。\\

标准库的I/O类型在3个头文件中:

\begin{enumerate}
	\item <iostream>定义了读写流的基本类型。
	\item <fstream>定义了读写文件的类型。
	\item <sstream>定义了读写string对象的类型。
\end{enumerate}

<fstream>中定义了3个I/O类来读写文件:

\begin{enumerate}
	\item ifstream从给定文件读数据。
	\item ofstream向给定文件写数据。
	\item fstream可读写文件。
\end{enumerate}

\vspace{0.5cm}

\subsection{文件打开模式}

每个流都有一个关联的文件模式,在打开文件时可以指定文件模式。

\begin{table}[H]
	\centering
	\setlength{\tabcolsep}{5mm}{
		\begin{tabular}{|c|l|}
			\hline
			\textbf{打开模式} & \textbf{作用}                                   \\
			\hline
			ios::in           & 以读方式打开                                    \\
			\hline
			ios::out          & 以写方式打开                                    \\
			\hline
			ios::app          & 以追加方式打开                                  \\
			\hline
			ios::ate          & 打开文件定位到文件末尾                          \\
			\hline
			ios::trunc        & 如果文件存在,其内容将被截断,即把文件长度设为0 \\
			\hline
		\end{tabular}
	}
	\caption{文件打开模式}
\end{table}

\mybox{文件I/O}

\begin{lstlisting}[language=C++]
#include <iostream>
#include <fstream>

using namespace std;

int main() {
    string name;
    int id;
    cout << "Enter name: ";
    cin >> name;
    cout << "Enter id: ";
    cin >> id;
    
    ofstream out("info.txt");
    out << name << " " << id << endl;
    out.close();
    
    ifstream in("info.txt");
    in >> name >> id;
    in.close();
    
    cout << "name = " << name << ", id = " << id << endl;
    return 0;
}
\end{lstlisting}

\begin{tcolorbox}
	\mybox{运行结果}
	\begin{verbatim}
Enter name: Terry
Enter id: 979489
name = Terry, id = 979489
	\end{verbatim}
\end{tcolorbox}

\newpage